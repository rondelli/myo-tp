\pdfoutput=1
\pdfpageattr{/Group << /S /Transparency /I true /CS /DeviceRGB>>}
\documentclass[11pt, a4paper, pdftex]{article}

\usepackage[british,spanish,es-minimal,es-sloppy]{babel}
\usepackage{verbatim}

\usepackage[hscale=0.65, vscale=0.70, vmarginratio=1:1,verbose]{geometry}

\usepackage[utf8]{inputenc}
% \usepackage[latin1]{inputenc}
\usepackage{lmodern}
\usepackage[T1]{fontenc}

\usepackage[scaled=.80]{beramono}
%\usepackage{inconsolata}
%\usepackage{palatino}

\usepackage[safe,tone,extra]{tipa}
\usepackage{mflogo}

\usepackage{graphicx}
	\DeclareGraphicsRule{*}{mps}{*}{}

\usepackage{xcolor}
\usepackage{colortbl}

\usepackage{tikz} % no tiene opciones
	\usetikzlibrary{calc}
	\usetikzlibrary{arrows}
	\usetikzlibrary{backgrounds}
	\usetikzlibrary{decorations.pathmorphing}
	\usetikzlibrary{shapes.geometric}
	\tikzset{>=latex'}

\usepackage{pgfplots}
\usetikzlibrary{pgfplots.dateplot}

%\usepackage{gnuplot-lua-tikz} %para incluir gráficos de gnuplot usando el driver de TikZ

\usepackage{amsmath}
\usepackage{amssymb}
\usepackage{dsfont}
\usepackage{nicefrac}
\usepackage{mathrsfs}
\usepackage[Euler]{upgreek}
\usepackage[nointegrals]{wasysym}
\usepackage[version=3,arrows=pgf]{mhchem}
\usepackage[colorlinks,verbose]{hyperref}
	\hypersetup{bookmarksnumbered, bookmarksopen, bookmarksopenlevel=2}
	\hypersetup{pdfstartview=FitH}
	%\hypersetup{pdfpagelayout=TwoColumnRight}

\usepackage{booktabs}

\usepackage{listings}
	%\lstset{showspaces=true,showtabs=true,tab=\rightarrowfill} % DEBUGGING IT!
	\lstloadlanguages{Java,Python,[LaTeX]TeX}

	%\lstset{language=Java} %\lstset{language={[LaTeX]TeX}}
	\lstset{backgroundcolor=\color{blue!5}}
	\lstset{ % set number
		numbers=left,
		stepnumber=1,
		numberstyle=\tiny\color{black!50},
		numbersep=5pt
	}
	\lstset{
		%basewidth={0.65em,0.45em}, % ugly! see documentation for details!
		columns={[c]fullflexible}, % kind of acceptable ;)
		tabsize=5,
		showstringspaces=false,
		%basicstyle=\footnotesize\ttfamily,
		basicstyle=\ttfamily,
		keywordstyle=\bfseries\color{blue!50!black},
		%identifierstyle=\color{blue!75},
		stringstyle=\color{green!50!black},
		commentstyle=\slshape\color{black!50},
		%mathescape=true;
		%morecomment=[is]{/*}{*/}
	}
	%\renewcommand{\lstlistingname}{Fitting}
	%\lstset{basicstyle=\footnotesize\ttfamily}
	%\lstset{numbers=left, numberstyle=\tiny, numbersep=6pt}
	%\lstset{backgroundcolor=\color{blue!10}}
	%\lstset{captionpos=b}
	%\lstset{xleftmargin=16pt}

\usepackage{listingsutf8}

\usepackage{float}

\newtheorem{definition}{Definición}[section]
\newtheorem{theorem}{Teorema}[section]
\newtheorem{examp}{Ejemplo}[section]
\newenvironment{example}{\begin{examp} \upshape}{$\blacklozenge$ \end{examp}}

\newtheorem{obs}{Observación}[section]
\newenvironment{observation}{\begin{obs} \upshape}{$\blacktriangle$ \end{obs}}

\newtheorem{algor}{Algoritmo}[section]
\newenvironment{algorithm}{\begin{algor} \upshape}{$\blacksquare$ \end{algor}}

\newtheorem{law}{Ley}[section]

\newenvironment{ok}{\renewcommand{\labelitemi}{$\surd$}\begin{itemize}}{\end{itemize}}
\newenvironment{not:ok}{\renewcommand{\labelitemi}{$\times$}\begin{itemize}}{\end{itemize}}

\newcommand{\eg}{e.g.~}
\newcommand{\ie}{i.e.~}

\newcommand{\Emph}[1]{\textsc{#1}}
%\newcommand{\Vec}[1]{\mathbf{#1}}
\newcommand{\oldstyle}[1]{\ensuremath{\mathnormal{#1}}}

\newcommand{\differential}[1]{\:d#1}
\newcommand{\Power}{\mathscr{P}}

% Unit vectors
\newcommand{\iunitvec}{\hat{\imath}}
\newcommand{\junitvec}{\hat{\jmath}}
\newcommand{\kunitvec}{\hat{k}}
\newcommand{\unitvec}[1]{\hat{#1}}
\newcommand{\Unitvec}[1]{\widehat{#1}}

\renewcommand{\labelenumi}{(\roman{enumi})}

\newcounter{BloodyArabic}
\newcommand{\RomanNumber}[1]{\setcounter{BloodyArabic}{#1}\textsc{\roman{BloodyArabic}}}
\newcommand{\rungnuplot}[1]{\immediate\write18{cd ../plots; gnuplot #1.gp && mpost -tex=latex #1.mp}}
\newcommand{\mtopgf}[1]{\immediate\write18{m4 #1.m4 | dpic -g > #1.tex}}

\def\Lucifer{Hernán Rondelli}
\def\Xime{Ximena Ebertz}
\def\Lu{Lucía Soria}
\def\Mati{Matías Curcio}

\title{
	\huge MyO -- Trabajo Práctico\\
}
\author{
	\Mati \and \Xime \and \Lucifer \and \Lu
}
\date{\small $\text{2}^{\text{do}}$ Semestre 2024
}

\flushbottom
\begin{document}

\renewcommand{\contentsname}{Contenidos}
\renewcommand{\listfigurename}{Listado de Figuras}
\renewcommand{\listtablename}{Listado de Tablas}
\renewcommand{\tablename}{Tabla}

\newcommand{\minimize}{\texttt{Minimize}\quad\,\,}
\newcommand{\maximize}{\texttt{Maximize}\quad\,\,}
\newcommand{\subjto}{\texttt{Subject to}\quad}

\pdfbookmark{Título}{title}\maketitle

\tableofcontents

\section{Primera parte}

Se deben realizar backups de $n$ archivos de la empresa \emph{BigData}
en discos idénticos de capacidad $d$, sin dividir los archivos. Cada
archivo es menor que la capacidad de un disco. Se necesita saber cuál es
la cantidad mínima de discos necesarios para almacenar todos los
archivos, considerando la capacidad de cada disco y el tamaño de cada
archivo.

\subsection{Modelo Lineal}

$F$: conjunto de nombres de $i$ archivos, con $i = 1, \ldots, n$. \\

$s_{i}$: tamaño del archivo $i$ en MB, con $i = 1, \ldots, n$. \\

$n$: cantidad de archivos. \\

$m$: cantidad de discos máxima necesaria, en principio $m = n$. \\

$D$: conjunto de $j$ discos de tamaño $d$, con $j = 1, \ldots, m$. \\

$d$: tamaño del disco en GB. \\ 

$x_{ij}$ binaria: $1$ indica si el archivo $i$ está en el disco $j$, $\forall i = \{1, \ldots, n\},\ \forall j = \{1, \ldots, m\}$, donde $x_{ij} \in {0, 1}$, $0$ en el caso de que no esté\\

$y_{j}$ binaria: $1$ indica que se está usando el disco $j$.

\begin{alignat*}{2}
	\minimize
	& \sum_{j = 1}^{m} y_{j}\\
	\subjto
	& \sum_{j = 1}^{m} x_{ij} = 1, \ \forall i \in \{1, \ldots, n\} \quad\text{(el archivo $i$ debe estar en un solo disco)}\\
	& \sum_{i = 1}^{n} s_{i} x_{ij} \le d\cdot y_{j}, \ \forall j \in \{1, \ldots, m\} \quad\text{(archivos que entran en el disco $j$)}\\
	% & x_{ij} \le y_{j}M \qquad\text{(no se puede usar un disco vacío)}\\
	& x_{ij} \in \{0,1\}, \quad \forall i \in \{1, \ldots, n\}, \ \forall j \in \{1, \ldots, m\}\\
	& y_{j} \in \{0,1\}, \quad \forall j \in \{1, \ldots, m\}
\end{alignat*}

\newpage
\subsection{Modelo Zimpl}

\lstinputlisting[inputencoding=utf8/latin1]{primera-parte/big-data.zpl}

\newpage
\subsection{Modelo PySCIPOpt}

\lstinputlisting[inputencoding=utf8/latin1,language=Python]{primera-parte/configuracion/generador_configuracion.py}

\lstinputlisting[inputencoding=utf8/latin1,language=Python]{primera-parte/configuracion/generardor_output.py}

\lstinputlisting[inputencoding=utf8/latin1,language=Python]{primera-parte/configuracion/leer_configuracion.py}

\lstinputlisting[inputencoding=utf8/latin1,language=Python]{primera-parte/distribuidor_archivos.py}

\lstinputlisting[inputencoding=utf8/latin1,language=Python]{primera-parte/main.py}

\section{Segunda parte}

\section{Tercera parte}

\section{Conclusiones}

El peronismo es lo más grande que hay.

%\newpage
\begin{thebibliography}{99}
%%%%%%%%%%%%%%%%%%%%%%%%%%%%%%%%%%%%%%%%%%%%%%%%%%%%%%%%%%%%%%%%%%%%%%%%%
	\bibitem{koch}
	%author:
	Koch, T.\
	%date:
	(2024)
	%title:
	\emph{Zimpl User Guide},
	%place and publisher:
	Berlin: ZIB.
	%pages:
	% pp.~43--64.
%%%%%%%%%%%%%%%%%%%%%%%%%%%%%%%%%%%%%%%%%%%%%%%%%%%%%%%%%%%%%%%%%%%%%%%%%
\end{thebibliography}

\end{document}
