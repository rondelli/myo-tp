\pdfoutput=1
\pdfpageattr{/Group << /S /Transparency /I true /CS /DeviceRGB>>}
\documentclass[11pt, a4paper, pdftex]{article}

\usepackage[british,spanish,es-minimal,es-sloppy]{babel}
\usepackage{verbatim}

\usepackage[hscale=0.65, vscale=0.70, vmarginratio=1:1,verbose]{geometry}

\usepackage[utf8]{inputenc}
% \usepackage[latin1]{inputenc}
\usepackage{lmodern}
\usepackage[T1]{fontenc}

\usepackage[scaled=.80]{beramono}
%\usepackage{inconsolata}
%\usepackage{palatino}

\usepackage[safe,tone,extra]{tipa}
\usepackage{mflogo}

\usepackage{graphicx}
	\DeclareGraphicsRule{*}{mps}{*}{}

\usepackage{xcolor}
\usepackage{colortbl}

\usepackage{tikz} % no tiene opciones
	\usetikzlibrary{calc}
	\usetikzlibrary{arrows}
	\usetikzlibrary{backgrounds}
	\usetikzlibrary{decorations.pathmorphing}
	\usetikzlibrary{shapes.geometric}
	\tikzset{>=latex'}

\usepackage{pgfplots}
\usetikzlibrary{pgfplots.dateplot}

%\usepackage{gnuplot-lua-tikz} %para incluir gráficos de gnuplot usando el driver de TikZ

\usepackage{amsmath}
\usepackage{amssymb}
\usepackage{dsfont}
\usepackage{nicefrac}
\usepackage{mathrsfs}
\usepackage[Euler]{upgreek}
\usepackage[nointegrals]{wasysym}
\usepackage[version=3,arrows=pgf]{mhchem}
\usepackage[colorlinks,verbose]{hyperref}
	\hypersetup{bookmarksnumbered, bookmarksopen, bookmarksopenlevel=2}
	\hypersetup{pdfstartview=FitH}
	%\hypersetup{pdfpagelayout=TwoColumnRight}

\usepackage{booktabs}

\usepackage{listings}
	%\lstset{showspaces=true,showtabs=true,tab=\rightarrowfill} % DEBUGGING IT!
	\lstloadlanguages{Java,[LaTeX]TeX}

	\lstset{language=Java} %\lstset{language={[LaTeX]TeX}}
	\lstset{backgroundcolor=\color{blue!5}}
	\lstset{ % set number
		numbers=left,
		stepnumber=1,
		numberstyle=\tiny\color{black!50},
		numbersep=5pt
	}
	\lstset{
		%basewidth={0.65em,0.45em}, % ugly! see documentation for details!
		columns={[c]fullflexible}, % kind of acceptable ;)
		tabsize=5,
		showstringspaces=false,
		%basicstyle=\footnotesize\ttfamily,
		basicstyle=\ttfamily,
		keywordstyle=\bfseries\color{blue!50!black},
		%identifierstyle=\color{blue!75},
		stringstyle=\color{green!50!black},
		commentstyle=\slshape\color{black!50},
		%mathescape=true;
		%morecomment=[is]{/*}{*/}
	}
	%\renewcommand{\lstlistingname}{Fitting}
	%\lstset{basicstyle=\footnotesize\ttfamily}
	%\lstset{numbers=left, numberstyle=\tiny, numbersep=6pt}
	%\lstset{backgroundcolor=\color{blue!10}}
	%\lstset{captionpos=b}
	%\lstset{xleftmargin=16pt}

\usepackage{float}

\newtheorem{definition}{Definición}[section]
\newtheorem{theorem}{Teorema}[section]
\newtheorem{examp}{Ejemplo}[section]
\newenvironment{example}{\begin{examp} \upshape}{$\blacklozenge$ \end{examp}}

\newtheorem{obs}{Observación}[section]
\newenvironment{observation}{\begin{obs} \upshape}{$\blacktriangle$ \end{obs}}

\newtheorem{algor}{Algoritmo}[section]
\newenvironment{algorithm}{\begin{algor} \upshape}{$\blacksquare$ \end{algor}}

\newtheorem{law}{Ley}[section]

\newenvironment{ok}{\renewcommand{\labelitemi}{$\surd$}\begin{itemize}}{\end{itemize}}
\newenvironment{not:ok}{\renewcommand{\labelitemi}{$\times$}\begin{itemize}}{\end{itemize}}

\newcommand{\eg}{e.g.~}
\newcommand{\ie}{i.e.~}

\newcommand{\Emph}[1]{\textsc{#1}}
%\newcommand{\Vec}[1]{\mathbf{#1}}
\newcommand{\oldstyle}[1]{\ensuremath{\mathnormal{#1}}}

\newcommand{\differential}[1]{\:d#1}
\newcommand{\Power}{\mathscr{P}}

% Unit vectors
\newcommand{\iunitvec}{\hat{\imath}}
\newcommand{\junitvec}{\hat{\jmath}}
\newcommand{\kunitvec}{\hat{k}}
\newcommand{\unitvec}[1]{\hat{#1}}
\newcommand{\Unitvec}[1]{\widehat{#1}}

\renewcommand{\labelenumi}{(\roman{enumi})}

\newcounter{BloodyArabic}
\newcommand{\RomanNumber}[1]{\setcounter{BloodyArabic}{#1}\textsc{\roman{BloodyArabic}}}
\newcommand{\rungnuplot}[1]{\immediate\write18{cd ../plots; gnuplot #1.gp && mpost -tex=latex #1.mp}}
\newcommand{\mtopgf}[1]{\immediate\write18{m4 #1.m4 | dpic -g > #1.tex}}

\def\Lucifer{Hernán Rondelli}
\def\Xime{Ximena Ebertz}
\def\Lu{Lucía Soria}
\def\Mati{Matías Curcio}

\title{
	\huge MyO -- Trabajo Práctico\\
}
\author{
	\Xime \and \Mati \and \Lucifer \and \Lu
}
\date{\small $\text{2}^{\text{do}}$ Semestre 2024
}

\flushbottom
\begin{document}

\renewcommand{\contentsname}{Contenidos}
\renewcommand{\listfigurename}{Listado de Figuras}
\renewcommand{\listtablename}{Listado de Tablas}
\renewcommand{\tablename}{Tabla}

\newcommand{\minimize}{\texttt{Minimize}\quad\,\,}
\newcommand{\maximize}{\texttt{Maximize}\quad\,\,}
\newcommand{\subjto}{\texttt{Subject to}\quad}

\pdfbookmark{Título}{title}\maketitle

\section*{Introducción}

\section*{Solución}

$F$: conjunto de nombres de $i$ archivos, con $i = 1, \ldots, n$. \\

$s_{i}$: tamaño del archivo $i$ en MB, con $i = 1, \ldots, n$. \\

$n$: cantidad de archivos. \\

$m$: cantidad de discos máxima necesaria, en principio $m = n$. \\

$D$: conjunto de $j$ discos de tamaño $d$, con $j = 1, \ldots, m$. \\

$d$: tamaño del disco en GB. \\ 

$x_{ij}$ binaria: $1$ indica si el archivo $i$ está en el disco $j$, $\forall i = \{1, \ldots, n\},\ \forall j = \{1, \ldots, m\}$, donde $x_{ij} \in {0, 1}$, $0$ en el caso de que no esté\\

$y_{j}$ binaria: $1$ indica que se está usando el disco $j$.

\begin{alignat*}{2}
	\minimize
	& \sum_{j = 1}^{m} y_{j}\\
	\subjto
	& \sum_{j = 1}^{m} x_{ij} = 1, \ \forall i \in \{1, \ldots, n\} \quad\text{(el archivo $i$ debe estar en un solo disco)}\\
	& \sum_{i = 1}^{n} s_{i} x_{ij} \le d\cdot y_{j}, \ \forall j \in \{1, \ldots, m\} \quad\text{(archivos que entran en el disco $j$)}\\
	& x_{ij} \le y_{j}M \qquad\text{(no se puede usar un disco vacío)}\\
	& x_{ij} \in \{0,1\}, \quad \forall i \in \{1, \ldots, n\}, \ \forall j \in \{1, \ldots, m\}\\
	& y_{j} \in \{0,1\}, \quad \forall j \in \{1, \ldots, m\}
\end{alignat*}

\section*{Conclusiones}

%\newpage
\begin{thebibliography}{99}
%%%%%%%%%%%%%%%%%%%%%%%%%%%%%%%%%%%%%%%%%%%%%%%%%%%%%%%%%%%%%%%%%%%%%%%%%
	%% \bibitem{acosta}
	%author:
	%% Acosta, F.\
	%date:
	%% (2012)
	%title:
	%% `La Escuela Secundaria Argentina en perspectiva histórica y comparada: Modelos institucionales y desgranamiento durante el siglo \textsc{xx}',
	%Journal:
	%% \emph{Cadernos de História da Educação},
	%pages:
	%% vol.~11, no.~1, pp.~131--144.
%%%%%%%%%%%%%%%%%%%%%%%%%%%%%%%%%%%%%%%%%%%%%%%%%%%%%%%%%%%%%%%%%%%%%%%%%
	\bibitem{durkhein}
	%author:
	Durkheim, E.\
	%date:
	(1996) %1922
	%title:
	\emph{Educación y sociología},
	%place and publisher:
	Barcelona: Ediciones Península,
	%pages:
	pp.~43--64.
%%%%%%%%%%%%%%%%%%%%%%%%%%%%%%%%%%%%%%%%%%%%%%%%%%%%%%%%%%%%%%%%%%%%%%%%%
	\bibitem{dussel}
	%author:
	Dussel, I.\ y Caruso, M.\
	%date:
	(1999)
	%title:
	\emph{La intervención del aula: Una genealogía de las formas de enseñar},
	%place and publisher:
	Buenos Aires: Santillana,
	%pages:
	pp.~89--136.
%%%%%%%%%%%%%%%%%%%%%%%%%%%%%%%%%%%%%%%%%%%%%%%%%%%%%%%%%%%%%%%%%%%%%%%%%
	%% \bibitem{heller}
	%author:
	%% Heller, Á.\ y Ferenc, F.\
	%date:
	%% (1994)
	%title:
	%% \emph{El péndulo de la modernidad: Una lectura de la era moderna después de la caída del comunismo},
	%place and publisher:
	%% Barcelona: Ediciones Península,
	%pages:
	%% pp.~128--155.
%%%%%%%%%%%%%%%%%%%%%%%%%%%%%%%%%%%%%%%%%%%%%%%%%%%%%%%%%%%%%%%%%%%%%%%%%
	\bibitem{naro}
	%author:
	Narodowski, M.\
	%date:
	(1999)
	%title:
	\emph{Después de clase: Desencantos y desafíos de la escuela actual},
	%place and publisher:
	Buenos Aires: Novedades Educativas.
%%%%%%%%%%%%%%%%%%%%%%%%%%%%%%%%%%%%%%%%%%%%%%%%%%%%%%%%%%%%%%%%%%%%%%%%%
	%% \bibitem{terigi}
	%author:
	%% Terigi, F.\
	%date:
	%% (2008)
	%title:
	%% `Los cambios en el formato de la escuela secundaria argentina: por qué son necesarios, por qué son  tan difíciles',
	%Journal:
	%% \emph{Propuesta Educativa},
	%pages:
	%\colorbox{red}{vol.~11}, no.~29, pp.~63--71.
	%% no.~29, pp.~63--71.
%%%%%%%%%%%%%%%%%%%%%%%%%%%%%%%%%%%%%%%%%%%%%%%%%%%%%%%%%%%%%%%%%%%%%%%%%
	\bibitem{varela}
	%author:
	Varela, J.\
	%date:
	(2009)
	%title:
	\emph{Educación (Sociología de la): Algunos modelos críticos},
	%place and publisher:
	Madrid: Editorial Plaza y Valdéz.
	%pages:
	%pp.~10--47.
%%%%%%%%%%%%%%%%%%%%%%%%%%%%%%%%%%%%%%%%%%%%%%%%%%%%%%%%%%%%%%%%%%%%%%%%%
	\bibitem{viñao}
	%author:
	Viñao, A.\
	%date:
	(2002)
	%title:
	\emph{Sistemas educativos, culturas escolares y reformas: Continuidades y cambios},
	%place and publisher:
	Madrid: Ediciones Morata,
	%pages:
	pp.~15--29, 44--69.
%%%%%%%%%%%%%%%%%%%%%%%%%%%%%%%%%%%%%%%%%%%%%%%%%%%%%%%%%%%%%%%%%%%%%%%%%
\end{thebibliography}

\end{document}
